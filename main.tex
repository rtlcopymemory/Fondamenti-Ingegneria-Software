\documentclass[10pt]{article}

\usepackage[italian]{babel}
\selectlanguage{italian}

\usepackage{scrextend} % indentation

\usepackage{graphicx} % for pics
\graphicspath{{images/}{../images/}} % source folder for pics

\usepackage{amsmath} % Math

\usepackage{hyperref} % Clickable Table of content
\hypersetup{
    colorlinks,
    citecolor=black,
    filecolor=black,
    linkcolor=black,
    urlcolor=black
}

\usepackage{tabularx} % Tables

\usepackage[margin=1.5in,bottom=1.5in,top=1.5in]{geometry} % changes the margin
% Fancy header and footer with page number on the right
\usepackage{fancyhdr}
\usepackage{lastpage}
\pagestyle{fancy}
\fancyfoot{}
\rfoot{\thepage}

% Algorithms Pseudo-code
\usepackage{algorithm}
\usepackage{algorithmic}

% Enviroment called theorem, resetting every section, printing 'Theorem'
\newtheorem{theorem}{Teorema}[section]
\newtheorem{definition}{Definizione}[section]
\newtheorem{demonstration}{Dimostrazione}[definition]

\def\code#1{\texttt{#1}}

\usepackage{subfiles} % Best loaded last in the preamble

\title{Fondamenti di Ingegneria del Software\\ \large Riassunto - TL;DR}
\author{Crippa Federico}
\date{A.A. 2020/2021}

\begin{document}

\maketitle
\tableofcontents

\newpage
\section*{Disclaimer}
Questo documento non \`e un sostituto alle lezioni ma solo un indice per il ripasso. Il contenuto sorvola alcuni aspetti che potrebbero essere utili o interessanti e non include dettagli o spiegazioni grafiche (ad esempio la sintassi di UML).

\vspace{1in}
\section*{Donations} \label{sec:Donations}
Questi appunti sono disponibili pubblicamente e gratuitamente presso il mio \underline{\href{https://github.com/WolfenCLI/Database-class-notes}{GitHub}}\\
Se ti sono state particolarmente utili, considera una donazione! \underline{\href{https://ko-fi.com/wolfencli}{Ko-Fi}}

\vspace{1in}
\section*{Non Licensing}
Questo documento \`e disponibile gratuitamente e chiunque \`e libero di contribuire, modificare o vendere copie di questo documento. Non viene applicata nessuna licenza siccome non ho trovato nessuna licenza con un ideale simile a GNU GPL.\\
Se si volesse contribuire, vedere la sezione sopra.

\newpage
\section{Modelli di sviluppo Plan-driven}
\subfile{sections/modelli-plan.tex}

\section{Modelli di sviluppo Agile}
\subfile{sections/modelli-agile.tex}

\newpage
\section{Requirements Engineering}
\subfile{sections/reqeng.tex}

\newpage
\section{Use case}
\subfile{sections/usecase.tex}

\section{High Level Design}
\subfile{sections/design.tex}

\section{Stili Architetturali}
\subfile{sections/stiliArch.tex}

\section{Stili di controllo}
\subfile{sections/stiliContr.tex}

\newpage
\section{Design By Contract}
\subfile{sections/descon.tex}

\newpage
\section{Principi di progettazione}
\subfile{sections/princ.tex}

\section{UML}
\subfile{sections/uml.tex}

\section{Framework vs Libreria}
\subfile{sections/framework.tex}

\newpage
\section{Design Patterns}
\subfile{sections/designPattern.tex}

\newpage
\section{Refactoring}
\subfile{sections/refactor.tex}

\newpage
\section{Persistenza}
\subfile{sections/persist.tex}

\section{Software Testing}
\subfile{sections/testing.tex}

\section{Extreme Programming}
\subfile{sections/XP.tex}

\end{document}
