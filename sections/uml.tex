Modi di utilizzo:
\begin{itemize}
    \item \textbf{Abbozzo/Sketch}: aiutare la comunicazione e la discussione
    \item \textbf{Blueprint}: fornire un modello completo/dettagliato da implementare
    \item \textbf{Linguaggio di programmazione}: fornire un “modello eseguibile”
\end{itemize}

\noindent \textbf{Modello di dominio}: \`E una rappresentazione visuale di classi concettuali o di oggetti del mondo reale di un dominio

\vspace{2mm}
\noindent In UML, un sistema viene descritto utilizzando diverse viste:\\
\textbf{Vista}: particolare aspetto di un Sistema dal punto di vista di uno specifico ruolo\\
Ogni Diagramma UML descrive una sola vista. Chiameremo Modello un insieme di diagrammi che descrivono aspetti diversi dello stesso sistema.

\vspace{2mm}
UML può essere usato in qualsiasi delle seguenti fasi: Requirements, Architecture, Design.
\begin{itemize}
    \item Requirements
    \begin{itemize}
        \item UML class diagrams
        \item UML use case diagrams
        \item UML activity diagram
        \item UML sequence diagrams
    \end{itemize}
    
    \item Architecture
    \begin{itemize}
        \item UML components diagram
        \item UML deployment diagram
    \end{itemize}
    
    \item Design
    \begin{itemize}
        \item UML class diagrams
        \item UML sequence diagrams
        \item UML communication diagrams
        \item UML object diagrams
        \item UML package diagrams
        \item UML state machine diagrams
        \item UML activity diagrams
    \end{itemize}
\end{itemize}

\subsection{Class Diagram}
Utilizzabile in 2 prospettive:
\begin{itemize}
    \item \textbf{Prospettiva concettuale}: Descriviamo gli elementi del “pezzo di mondo” che ci interessa modellare. Classe UML = concetto proprio del dominio
    \item \textbf{Prospettiva software}: Descriviamo il design di un software, ovvero i moduli software che costituiranno l’implementazione vera e propria del sistema. Classe UML = classe in un linguaggio OO
\end{itemize}

\subsubsection{Linguaggio OCL per Vincoli}
\textbf{Context}: $<$Classe$>$\\
\textbf{Inv$|$Pre$|$Post}: $<$Condizione$>$

\vspace{4mm}
\noindent Le condizioni possono anche contenere:
\begin{itemize}
    \item \textbf{@time}: Esempio. self.età = self.età@pre + 1
    \item \textbf{forAll()}: Esempio. self.parco$\rightarrow$forAll(v$|$v.colore=nero)
\end{itemize}

\subsection{Sequence Diagram}
I diagrammi di interazione descrivono la collaborazione di un gruppo di oggetti. Ovvero lo scambio di messaggi.

\vspace{2mm}
\noindent \textbf{Dove e come utilizzarli}:
\begin{itemize}
    \item Sono un ottimo mezzo per visualizzare l’interazione tra oggetti, non la logica di controllo.
    \item Stabilire con precisione quali sono gli eventi di input (operazioni di input e parametri) e di output (dati di ritorno) del sistema
\end{itemize}

\break
\subsection{State Machine Diagram}
Le state machine vengono usate per descrivere il comportamento di una entità come variazione del suo stato interno quando è sottoposta a sollecitazioni dal mondo esterno.

\vspace{2mm}
\noindent \textbf{Dove e come usarle}
\begin{itemize}
    \item Usare i diagrammi degli stati solo per le entità che hanno una logica interna interessante e complessa
    \begin{itemize}
        \item Oggetti e sistemi di controllo
        \item Distributori automatici
        \item Sistemi di gestione documentale
        \item GUI
    \end{itemize}
\end{itemize}

\subsection{Activity Diagram}
Descrivono come viene svolta un’attività relativa ad una qualsiasi entità. Usati in contesti diversi, fasi diverse dello sviluppo e per scopi diversi.

\vspace{2mm}
\noindent \textbf{Dove e come usarle}
\begin{itemize}
    \item Modellare processi di business e workflow (analisi)
    \item Modellare il flusso di un caso d’uso (analisi)
    \item Modellare un’operazione di una classe (progettazione)
    \item Modellare un algoritmo (progettazione)
    \item Come linguaggio di programmazione (codifica)
\end{itemize}

\subsection{Component Diagram}
I component Diagrams sono molto diversi dagli altri Diagrams. Mentre gli altri rappresentano il sistema, i Component Diagrams sono usati per descrivere il funzionamento e il comportamento di varie componenti di un sistema.

\vspace{2mm}
\noindent \textbf{Dove e come usarli}
\begin{itemize}
    \item Per rappresentare componenti di un sistema a runtime.
    \item Aiutano durante il testing
    \item Visualizza la connessione fra varie componenti
\end{itemize}

\break
\subsection{Deployment Diagram}
Mostra la relazione tra hardware e software in un sistema e ha un forte link con il Component diagram.\\
\`E importante specificare gli \textbf{artefatti}: Sono entità concrete del mondo reale che vengono scambiante fra le varie componenti. (Es. zip, dati, codice sorgente, files in generale).

\vspace{2mm}
\noindent \textbf{Dove e come usarli}
\begin{itemize}
    \item Modellare la topologia di una rete di sistemi
    \item Modellare un sistema distribuito
    \item Forward e Reverse Engineering
\end{itemize}

\subsection{Package Diagram}
Un package (in UML) è un costrutto che permette di prendere un numero arbitrario di elementi UML (classi, casi d’uso, modelli UML) e raggrupparli assieme.

\vspace{2mm}
\noindent \textbf{Dove e come usarli}
\begin{itemize}
    \item Per semplificare e schematizzare class diagram complessi
\end{itemize}