\`E una procedura sistematica che prevede l’esecuzione di un sistema software (SUT) con l’intento di trovare failure (e poi il fault associato)\\
Il \textbf{Software testing} rivela i \textbf{failure}, il \textbf{Debugging} li identifica e rimuove.\\
Non \`e possibile testare esattamente ogni possibile input.

\subsection{Debugging}
Due fasi:
\begin{enumerate}
    \item fault localization/location
    \item fault removal
\end{enumerate}

\subsection{Testing}
Essitono due categorie
\begin{enumerate}
    \item White Box (o structural testing): basato su una conoscenze esplicita del SUT e della sua struttura
    \item Black box (o functional testing): NON basato su una conoscenze esplicita del SUT e della sua struttura
\end{enumerate}

\noindent Eseguiti in vari fasi con nomi e scopi diversi:
\begin{itemize}
    \item Testing di unità (implementazione): Testano le singole unità
    \item Testing di integrazione (implementazione/integrazione): Testano il sistema come insieme e l'integrazione delle unità fra di loro
    \item Testing di sistema (integrazione di sistema): Testano che il software soddisfi i requisiti e i bisogni dell'utente
    \item Testing di regressione (manutenzione): verificano che non ci siano stati side effect dopo il cambiamento di una parte di codice (ad esempio bug fix)
\end{itemize}