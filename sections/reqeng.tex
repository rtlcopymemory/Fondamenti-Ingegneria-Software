Possono essere scritti o per \textbf{il cliente} o per \textbf{gli sviluppatori}.\\
Per ogni modello di sviluppo ci sono modi diversi di rappresentare i requisiti, in generale, più un modello \`e plan driven e più la rappresentazione \`e formale mentre più \`e agile più sono rappresentati in maniera rapida e informale (sticky notes).\\
Lo scopo del requirements engineering \`e di creare e tener aggiornato un documento che specifica le funzionalità e i servizi del sistema da produrre.

\subsection{Requisiti funzionali o non funzionali}
\textbf{Funzionali}:
Descrivono le funzionalità ed i servizi che saranno forniti dal sistema.\\
Indipendenti dall'implementazione di una soluzione.

\noindent \textbf{Non funzionali}:
Non sono collegati direttamente con le funzionalità implementate dal sistema, ma piuttosto alle modalità operative, di gestione, ...\\
Definiscono vincoli sul sistema e sullo sviluppo del sistema.\\
In generale riguardano la scelta di linguaggi, piattaforme, strumenti (tools), tecniche d’implementazione, ma anche: prestazioni, questioni etiche, ...

\subsection{Fasi}
\begin{enumerate}
    \item \textbf{Elicitation}: La raccolta di informazioni dal client.
    \item \textbf{Analisi dei requisiti}: Include l'eliminazione di requisiti contraddittori e la loro prioritizzazione.
    \item \textbf{Definizione e specifica}: definizione per il cliente e specifiche per i programmatori.
    \item \textbf{Validazione}: Convalida, spesso effettuata tramite formal peer review.
\end{enumerate}

\subsection{Proprietà}
\begin{itemize}
    \item Validità-correttezza
    \item Consistenza
    \item Completezza
    \item Realismo
    \item Inequivocabilità
    \item Verificabilità
    \item Tranciabilità
\end{itemize}