L’architettura di un sistema può conformarsi a uno stile architetturale. Conoscere gli stili architetturali può semplificare il problema di definire l’architettura software

\vspace{2mm}

\noindent \textbf{Elementi di uno stile}
\begin{itemize}
    \item \textbf{Tipo di componenti}
    \item \textbf{tipo di connettori}
\end{itemize}

\noindent \textbf{Stili}
\begin{itemize}
    \item Layered (Stratificato)
    \item Repository
    \item Client/server
    \item Two tier / three tier
    \item P2P
    \item Pipe and Filter
    \item Broadcast Model
    \item Service Oriented Architecture
    \item Microservice
\end{itemize}

\break
\subsection{Stratificato (Layered)}
Organizza il sistema in un insieme di livelli ognuno dei quali fornisce un insieme di servizi. Un livello “si appoggia” solo sul livello inferiore (cioè usa solo i servizi del livello inferiore).

\subsubsection{Vantaggi}
\begin{itemize}
    \item Se cambia l’interfaccia di un livello, solo il livello superiore ne è influenzato
    \item Supporta il riuso: differenti implementazioni dello stesso livello possono essere usate in modo intercambiabile
\end{itemize}

\subsubsection{Svantaggi}
\begin{itemize}
    \item Spesso però è artificiale strutturare il sistema in questo modo
    \item Può risultare inefficiente (pila di chiamate!) e spesso questo porta a “shortcut”
\end{itemize}

\subsection{Repository Model}
I dati condivisi sono mantenuti in un database centrale (repository) a cui hanno accesso tutti i sotto-sistemi.

\subsubsection{Vantaggi}
\begin{itemize}
    \item Modo efficiente di condividere grandi quantità di dati (non sono trasmessi esplicitamente tra sottosistemi)
    \item Gestione centralizzata di backup, security, etc.
\end{itemize}

\subsubsection{Svantaggi}
\begin{itemize}
    \item I sottosistemi devono accordarsi su un modello dei dati per il repository, necessità di compromesso
    \item L’evoluzione dello schema dei dati è difficile e costosa
    \item Difficile da rendere distribuito in maniera efficiente
\end{itemize}

\subsection{Client-Server Model}
Modello di sistema distribuito che mostra come i dati e la computazione possono essere distribuiti su server che forniscono i servizi e client che li usano.

\subsubsection{Client-Server a 2 livelli}
Diviso in:
\begin{itemize}
    \item Client
    \item Server che comprende logica e Database
\end{itemize}

\subsubsection{Client-Server a 3 livelli}
Diviso in:
\begin{itemize}
    \item Client
    \item Logica Server
    \item Database Server
\end{itemize}

\subsubsection{Vantaggi}
\begin{itemize}
    \item Possibile migliorare le performance di un sistema “spostando” la computazione tra client e server e viceversa
    \item Supporta il riuso: server che forniscono un servizio possono essere riusati in diverse applicazioni
    \item Scalabile: si aggiungono dei server
    \item Aggiunta di nuovi server o upgrade di quelli esistenti è semplice
\end{itemize}

\subsubsection{Svantaggi}
\begin{itemize}
    \item Mancanza di modello dei dati condiviso tra server, lo scambio di dati può essere inefficiente
    \item Operazioni di gestione ridondanti nei vari server
    \item Mancanza di registro centrale di nomi e servizi, può essere difficile determinare i server e servizi disponibili
\end{itemize}

\subsection{Peer-to-Peer Model}
Ogni componente esegue le funzioni sia di client che di server. Cosa distingue un peer da un altro sono i dati. I sistemi P2P scalano bene e sono fault-tolerant.

\subsubsection{Vantaggi}
\begin{itemize}
    \item Aggiungere un “peer” vuol dire aggiungere nuove capabilities (in termini di nuovi dati)
    \item I dati sono replicati su diversi peer per cui se un nodo viene sconnesso per una qualche ragione non si perde (molta) informazione
    \item Sono fault-tolerant
    \item Scalano bene
\end{itemize}

\break
\subsection{Pipe and Filter Model}
I filtri effettuano trasformazioni che elaborano i loro input per produrre output.\\
Le pipe sono connettori che trasmettono i dati tra filtro e filtro.

\subsubsection{Vantaggi}
\begin{itemize}
    \item Supporta il riuso delle trasformazioni
    \item Intuitivo: permette di capire il funzionamento del sistema come composizione dei filtri
    \item Aggiunta di nuove trasformazioni facile
    \item Abbastanza semplice da implementare
    \item Supporta esecuzione concorrente/parallela
\end{itemize}

\subsubsection{Svantaggi}
\begin{itemize}
    \item Non adatto per sistemi interattivi
    \item Di solito porta allo sviluppo di sistemi batch dove ogni filtro compie una completa trasformazione dell’input
    \item Lavoro extra per parsing e unparsing dei dati nei vari filtri
\end{itemize}

\subsection{Architetture Eterogenee}
Ci sono due modi di combinare Stili architetturali:
\begin{itemize}
    \item Modo gerarchico
    \item Permettendo che una componente sia un mix di architetture
\end{itemize}