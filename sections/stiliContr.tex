Esistono due stili di controllo implementati in modelli diversi:
\begin{itemize}
    \item \textbf{Controllo Centralizzato}: un sottosistema ha la responsabilità generale del controllo e “avvia e ferma” gli altri sottosistemi
    \begin{itemize}
        \item Call - return model
        \item Manager Model
    \end{itemize}
    
    \item \textbf{Controllo basato su eventi}: non esiste la figura di “controllore”; ogni sottosistema può rispondere ad eventi generati da altri sottosistemi
    \begin{itemize}
        \item Broadcast model
    \end{itemize}
\end{itemize}

\subsection{Call - Return Model}
Un sottosistema di controllo (Main) ha la responsabilità di gestire l’esecuzione degli altri sottosistemi.\\
Il Main decide l’ordine di esecuzione dei sottosistemi.\\
Modello top-down in cui il controllo parte dalla radice e si sposta verso il basso\\
Applicabile solo a sistemi sequenziali

\subsection{Manager Model}
Usato se i sottosistemi controllati possono funzionare in parallelo.\\
Una componente del sistema (Il Manager) controlla l’avvio, la terminazione e il coordinamento degli altri processi di sistema

\subsection{Broadcast Model}
Invece di invocare una procedura direttamente, un componente può annunciare (broadcast) uno o più eventi. Quindi tutti i componenti lo ricevono ma solo i componenti interessati lo elaborano.

\noindent \textbf{Vantaggi}:
\begin{itemize}
    \item Semplice aggiungere/togliere/sostituire componenti
    \item Semplice il riuso di componenti
\end{itemize}

\noindent \textbf{Svantaggi}:
\begin{itemize}
    \item I sottosistemi quando mandano un evento non sanno se e quando l’evento verrà gestito
    \item Scambio dei dati può complicare le cose (o passati con l'evento o utilizzo di una repository)
    \item Possono essere difficili da implementare e soprattutto testare
\end{itemize}