XP è un processo di sviluppo per Software OO ideato per piccoli gruppi (4-20) che cerca di mantenersi agile e flessibile.\\
Non usa linguaggi di analisi o design (UML), ma dalla raccolta dei requisiti, eseguita con le “User stories”, passa velocemente alla codifica.\\
Una \textbf{user story} è un sintetico “caso d’uso” che viene scritto in linguaggio naturale su una scheda.\\
Invece di un rilascio unico XP prevede frequenti rilasci di codice funzionante ogni 1-4 settimane.

\subsection{Design}
\textbf{Metodo CRC}: Si usano foglietti adesivi su cui vengono scritti il nome della classe, le responsabilità e le collaborazioni con le altre classi.\\
Per avere sempre ben chiaro il design è importante che il codice sia il più autodocumentato e “chiaro” possibile (meaningful code).

\subsection{Codifica}
Non si pianifica per il riuso, ma si sviluppa nel modo più semplice possibile.\\
Tutti i programmatori devono attenersi alle stesse “code conventions”.\\
Spesso fa uso del Pair Programming.

\subsection{Unit Testing}
Per ogni classe vanno scritti dei casi di test per verificarne il funzionamento.\\
I test vanno scritti prima del codice stesso perché aiutano nello sviluppo. (\textbf{TDD})

\subsection{Refactoring}
Per mantenere il codice semplice e meaningful occorre ‘ristrutturarlo’ spesso: Ogni volta che la classe supera i casi di test si applica il refactoring.\\
La ristrutturazione continua è possibile perchè si dispone dei casi di test di unità.

\subsection{Acceptance Testing}
Sono previsti dei test funzionali di accettazione, basati sulle “User stories” e concordati con il cliente. Durante lo sviluppo possono anche non essere tutti passati.\\
La percentuali di test passati indica il prograsso dello sviluppo.